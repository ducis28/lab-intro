\documentclass[uplatex,jis2004,a4paper,11pt]{jsarticle}
\usepackage[deluxe]{otf}

%%% 余白設定
%\usepackage[top=3cm, bottom=3cm, left=2cm, right=2cm]{geometry}
\usepackage[driver=dvipdfm,truedimen]{geometry}
\geometry{left=15truemm,right=15truemm,top=15truemm,bottom=20truemm}

%%% フォント設定
\usepackage[alphabet,unicode]{pxchfon}
\setminchofont[0]{UDDigiKyokashoN-R.ttc}% UDデジタル教科書体N-R
\setboldminchofont[0]{UDDigiKyokashoN-B.ttc}% UDデジタル教科書体N-B

\usepackage{amssymb,amsmath}
%\usepackage[dvipdfmx,hidelinks]{hyperref}
\usepackage{hyperref}

\usepackage{enumitem}
\setlist{leftmargin=5mm}
\renewcommand{\labelitemi}{$\triangleright$}
\renewcommand{\labelitemii}{$\blacktriangleright$}

\usepackage{multicol}

%%% section のフォント設定
\usepackage{titlesec}
\titleformat{\section}{\Large\bfseries}
  {\thesection}{1em}
  {}
  [\titlerule]
\titleformat*{\subsection}{\Large\bfseries}
\titleformat*{\subsubsection}{\large\bfseries}

\renewcommand{\emph}[1]{\textbf{#1}}

\pagestyle{empty}

\title{計算情報学研究室}
%\author{深川 大路 (Fukagawa, Daiji)}
\date{}



\begin{document}

\begin{minipage}[t]{.45\textwidth}
\noindent
{\LARGE\bfseries\rmfamily {\Huge㉘} 計算情報学研究室}\\
\hspace{.4cm} Computational Informatics Laboratory
\end{minipage}
\begin{minipage}[t]{.45\textwidth}
\begin{tabular}{l@{ }l}
担当教員: & 深川 大路 (FUKAGAWA Daiji) \\
e-mail: & \texttt{dfukagaw@mail.doshisha.ac.jp} \\
研究室: & MK515 (教員), MK211 (学生)
\end{tabular}
\end{minipage}

%\maketitle

\section{教員について}

専門分野は\emph{アルゴリズムと計算量の理論},特に,\emph{離散データのマッチング}です。
ざっくりと説明すると,アルゴリズムとは問題の解き方であり,計算量はそのアルゴリズムの効率を表す尺度のひとつであり,離散データは,文字列や木などの単純な構造を持つデータから,グラフとよばれる複雑なネットワーク構造を持つデータを含むものを指します。

アルゴリズムを研究する最大の目的は,計算機 (computer) の処理を速くすることです。そのための手段として,アルゴリズムの計算量や問題の複雑さを数学的に解析したり,実際に計算機プログラムを実装したりといった作業を行います。

一方で,講義・研究・その他の用務上の必要からプログラムを書くことも多く,その関係から,\emph{情報システムの開発}にも大いに興味があります。最近はこちらに重点を置くことが多くなっています。利用するプログラミング言語は最近では Python と JavaScript が多く,他にも C/C++, Java, Ruby などを使います。

\section{卒業研究について}

卒業研究のテーマは,配属後なるべく自分自信で考えてもらいます。ただし何でもありではなく,教員とディスカッションを重ね,取り組むに値するかどうかを吟味して決定します。その過程において,しっかりと先行研究を調査してください。教員から批判的な意見も投げかけますので,やりたいこと・考えたこと・調べたことをきちんと説明してください。ここで手を抜くと後で苦労した挙句に行き詰まり,泣く泣くテーマを変えるということになりかねません。テーマの種を見極める作業は意外と重要です。
教員自身が抱えているテーマが自分の興味にマッチしていれば,それを選んでいただくこともオススメです。教員はモチベーションが上がります。ただし実際に卒業研究を行う皆さんにとってのモチベーションがないとお互いに困ります。

\subsection*{過去の卒業研究題目(過去4期分)}
\vspace{-.5cm}
\begin{multicols}{2}
{\fontsize{9pt}{0mm}\selectfont
\subsubsection*{2020年度}
\begin{itemize}
  \item プロ野球における外国人選手の役割---日本野球機構成績データを通して---
  \item インタラクティブ制約付きクラスタリングにおける初期クラスタ精度の影響について
  \item ジャンルごとに頻出するbigramを考慮した古典籍の文字認識
  \item 詰めアルゴ問題の自動作成
  \item ダムの流入量・放流量を考慮した洪水時の河川水位予測
  \item 生活習慣等を特徴量とした若年層のうつ傾向の予測
  \item 第二外国語暗記学習のための替え歌生成システムの提案
  \item ピクトグラムアイコンを用いたラグビー観戦支援システムの提案
  \item トーナメントモデルを用いた属性と意見の対の抽出
  \item クロンダイクにおける成功率計算のための指針の提案
  \item 計算機を用いた和歌の基礎分析---句と音韻に着目して---
  \item UCTの探索空間を制限することによる勝敗への影響について---花合わせを題材にして---
  \item 逆翻訳によるデータ拡張を利用した古文・現代文のニューラル機械翻訳
  \item 暗記学習における絵カルタの有用性
  \item シャーロック・ホームズシリーズの物語パターン抽出
\end{itemize}
\vspace{-5mm}
\subsubsection*{2019年度}
\begin{itemize}
  \item 簡易宿所の戦略構築を支援するユーザレビュー分析におけるLDAの利用
  \item 機械学習におけるOptimizerのAccuracyの比較分析
  \item CNNとLSTMによる増減の2値分類予測の精度比較―為替データを用いて― 
\end{itemize}
\vspace{-5mm}
\subsubsection*{2018年度}
\begin{itemize}
  \item クロンダイクにおける成功可能性―探索の深さに着目して―
  \item バドミントンのゲームデータ分析―女子ダブルスを対象として―
  \item タイポグリセミア現象を用いた CAPTCHA の提案
  \item 字母の違いを考慮した機械学習によるくずし字認識
\end{itemize}
\vspace{-5mm}
\subsubsection*{2015年度}
\begin{itemize}
  \item ブルースに特化したドラムの伴奏付けシステムの開発
  \item AR技術を用いたギター練習のための楽譜情報動的提示システムの提案
  \item 梅田地下街におけるシミュレーションを用いた浸水被害予測
  \item 複数料理レシピにおける並行調理スケジュールの提案と評価
\end{itemize}
%\subsubsection*{2014年度}
%\begin{itemize}
%  \item マッシュアップ楽曲作成支援システムの構築に向けた基礎的検討
%    % 日本音響学会音楽音響研究会 2014年11月研究会
%  \item トランプゲーム大貧民において8切りの活用方法がプレイヤーの強さに与える影響
%    % 第9回 UECコンピュータ大貧民大会 ライト級 富豪 (第2位)
%\end{itemize}
\vspace{-.4cm}
}
\end{multicols}

\clearpage
\section{卒業研究の進め方}

卒業研究は個人で行います。
方向性や進捗を確認できるよう,\emph{全体ミーティング(毎週)}と\emph{個別ミーティング(随時)}を活用してください。
人数が少ない年度は個別ミーティング中心,人数が多い年度は全体ミーティング中心となります。
2020年度は卒研生を3つのゼミ(アルゴリズム,機械学習,情報システム)に分割してゼミ単位でミーティングを行いました。
2021年度は週2回の全体ミーティング中心です。
勉強会も開催しています(プログラミング,機械学習)。

研究室に配属が決まったら,遠慮せずに研究室に来てください。
2021年度は人数もコロナ定員に収まることから体面でミーティングを行っています。
先輩の様子を見て,話を聞いてみてください。卒業研究において必要なものは何か,考えてください。
見る・聞くだけでなく自分でやってみることも大切です。
\emph{積極性・主体性}を重視します。\emph{2年次生のうちから先輩の卒業研究を引き継ぎ,改善したものを学外で発表して受賞した先輩たち}もいます。ぜひチャレンジしてください。
\vspace{-.3cm}
\subsection*{3年次生}
\vspace{-.3cm}
\begin{itemize}
  \item 必修科目ジョイントリサーチで,卒業研究に必要な知識・技術・姿勢を身に付けます。
  \item 最新技術は日々更新されます。新しい技術を活用するには専門知識も重要ですが,基礎力を鍛えることを優先してください。
\end{itemize}
\vspace{-.6cm}
\subsection*{4年次生}
\vspace{-.3cm}
\begin{itemize}
  \item 春学期:研究計画書(初稿6月初旬,最終稿7月初旬),コロキアム発表練習
  \item 秋学期:コロキアム発表,卒業論文(初稿11月初旬,最終稿12月初旬),試問会(1月下旬)
\end{itemize}

\section{研究室配属希望者のための面接}

\begin{itemize}
  \item まず電子メールでアポイントメントをとってください。
    少なくとも一通目は,面識のない相手に送るビジネス文書のつもりで書いてみてください(その能力があることを見せてください)。
    メールには日程調整を円滑に行うために必要な情報を含めてください。
    面接の形式は体面・オンラインのいずれでも受け付けます。
  \item アポイントメントがとれたら面接までに準備をしてください。
    \begin{itemize}
      \item 〔オンラインの場合〕カメラ・マイク・ヘッドフォン(イヤフォン)の準備
      \item 〔オンラインの場合〕zoom ソフトウェアのインストール・更新,接続テスト
      \item 成績が確認できるもの(成績表の単位数欄のスクリーンショットなど)
    \end{itemize}
  \item 2021年度は,e-class の教員コンタクト受付システムへの申請も必要です。
  \item 教員コンタクト時に確認する内容:「卒業研究に対する意欲,準備状況 (これまでに学習したこと),今後の計画 (これから学習したいこと),最近注目している技術・ニュース,読んだ本などについて考えをまとめる」
\end{itemize}

\section{配属希望者へのメッセージ}

最も重要なことはマッチングです。面談での受け答えを重視します。とはいえ緊張し過ぎて自分を出せないのも困ります。自信を持ってあなたのことを十分に説明してください。

「必要であれば何でもやってみる」気持ちで広く勉強をしてください。その上で,ここが重要だというところは,とことん深く掘り下げてください。理解した「つもり」は危険です。
いまはできないこと,分からないことであっても,日々の努力の積み重ねで乗り越えて,
できること,分かることを貪欲に広げましょう。

「研究対象への深い興味」「研究の遂行に必要な知識を探し求める姿勢」
「卒業研究に真剣に取り組む誠実さ」「日常生活を楽しめる前向きな姿勢」
を備えた人を募集します。真剣に研究をしたい人を力の限り応援します!

%
%
\end{document}
