%\documentclass[11pt,a4paper]{jsarticle}
%\documentclass[a4paper,twocolumn,uplatex,dvipdfmx]{bxjsarticle}
\documentclass[a4paper,uplatex,dvipdfm]{bxjsarticle}

\usepackage{amsmath}
\usepackage[dvipdfmx,hidelinks]{hyperref}
\usepackage{pxjahyper}

\usepackage{pxbase}
\usepackage[bxutf8]{inputenc}
\usepackage[uplatex,deluxe,expert]{otf}

\usepackage{enumitem}
\setlist{leftmargin=5mm}

%%% title のフォント設定
\makeatletter
\renewcommand{\title}[1]{\gdef\@title{\bfseries\rmfamily#1}}
\makeatother

%%% section のフォント設定
%\makeatletter
%\renewcommand{\headfont}{\rmfamily\bfseries}
%\makeatother
%\usepackage{titlesec}
%\titleformat*{\section}{\large\bfseries}
%\renewcommand{\headfont}{\gtfamily\sffamily\bfseries}
\usepackage{titlesec}
\titleformat{\section}
  {\normalfont\rmfamily\Large\bfseries}
  {\thesection}{1em}
  {}
  [\titlerule]

%
%\setlength{\textwidth}{\fullwidth}
%\setlength{\textheight}{40\baselineskip}
%\addtolength{\textheight}{\topskip}
%\setlength{\voffset}{-0.2in}
%\setlength{\topmargin}{0pt}
%\setlength{\headheight}{0pt}
%\setlength{\headsep}{0pt}
\pagestyle{empty}
%
\title{計算情報学研究室}
%\author{深川 大路 (Fukagawa, Daiji)}
\date{}

\begin{document}

\begin{minipage}[t]{.45\textwidth}
\noindent
{\Huge\bfseries\rmfamily 計算情報学研究室}
\end{minipage}
\begin{minipage}[t]{.50\textwidth}
\begin{tabular}{ll}
担当教員: & 深川 大路 (FUKAGAWA Daiji) \\
e-mail: & \texttt{dfukagaw@mail.doshisha.ac.jp} \\
研究室: & MK515 (教員), MK211 (学生)
\end{tabular}
\end{minipage}

%\maketitle

\section{教員について}

専門分野は{\sffamily\bfseries アルゴリズムと計算量の理論},特に,{\sffamily\bfseries 離散データのマッチング}です.
ざっくりと説明すると,アルゴリズムとは問題の解き方であり,計算量はそのアルゴリズムの効率を表す尺度のひとつであり,離散データは,文字列や木などの単純な構造を持つデータから,グラフとよばれる複雑なネットワーク構造を持つデータを含むものを指します.

アルゴリズムを研究する最大の目的は,計算機 (computer) の処理を速くすることです.そのための手段として,アルゴリズムの計算量や問題の複雑さを数学的に解析したり,実際に計算機プログラムを実装したりといった作業を行います.

一方で,講義・研究・その他の用務上の必要からプログラムを書くことも多く,その関係から,{\sffamily\bfseries 情報システムの開発}にも大いに興味があります.最近はこちらに重点を置くことが多くなっています.利用するプログラミング言語は最近では Python と JavaScript が多く,他にも C/C++, Java, Ruby などを使います.

\section{卒業研究について}

卒業研究のテーマは,配属後なるべく自分自信で考えてもらいます.ただし何でもありではなく,教員とディスカッションを重ね,取り組むに値するかどうかを吟味して決定します.その過程において,しっかりと先行研究を調査してください.教員から批判的な意見も投げかけますので,やりたいこと・考えたこと・調べたことをきちんと説明してください.ここで手を抜くと後で苦労した挙句に行き詰まり,泣く泣くテーマを変えるということになりかねません.テーマの種を見極める作業は意外と重要です.
教員自身が抱えているテーマが自分の興味にマッチしていれば,それを選んでいただくこともオススメです.教員はモチベーションが上がります.ただし実際に卒業研究を行う皆さんにとってのモチベーションがないとお互いに困ります.

\subsection*{過去の卒業研究題目(過去4期分)}
\begin{minipage}[t]{.47\textwidth}
{\footnotesize
\subsubsection*{2018年度}
\begin{itemize}
  \item クロンダイクにおける成功可能性―探索の深さに着目して―
  \item バドミントンのゲームデータ分析―女子ダブルスを対象として―
  \item タイポグリセミア現象を用いた CAPTCHA の提案
  \item 字母の違いを考慮した機械学習によるくずし字認識
\end{itemize}
\subsubsection*{2015年度}
\begin{itemize}
  \item ブルースに特化したドラムの伴奏付けシステムの開発
  \item AR技術を用いたギター練習のための楽譜情報動的提示システムの提案
  \item 梅田地下街におけるシミュレーションを用いた浸水被害予測
  \item 複数料理レシピにおける並行調理スケジュールの提案と評価
\end{itemize}
}
\end{minipage}
\hspace{.03\textwidth}
\begin{minipage}[t]{.46\textwidth}
{\footnotesize
\subsubsection*{2014年度}
\begin{itemize}
  \item マッシュアップ楽曲作成支援システムの構築に向けた基礎的検討
    % 日本音響学会音楽音響研究会 2014年11月研究会
  \item トランプゲーム大貧民において8切りの活用方法がプレイヤーの強さに与える影響
    % 第9回 UECコンピュータ大貧民大会 ライト級 富豪 (第2位)
\end{itemize}
\vspace{-.4cm}
\subsubsection*{2013年度}
\begin{itemize}
  \item 指の上げ下げに頑健な指の種類認識
  \item 論文情報データベースを用いた研究トピックの可視化
  \item 日本一周最短鉄道経路をめざして
  \item 不完全情報ゲームにおける意思決定基準~こいこいを題材として~
  \item トランプゲーム大貧民における切札の重要性
  \item プレイヤーのスキルに応じたクリケットゲーム最適戦略の提案
  \item 大学生の食事マナーを規定する要因の検討―しつけと食事環境に着目して―
\end{itemize}
}
\end{minipage}

\section{卒業研究の進め方}

卒業研究は個人で行います.方向性や進捗を確認できるよう,{\sffamily\bfseries 全体ミーティング(毎週)}と{\sffamily\bfseries 個別ミーティング(随時)}を活用してください.人数が少ない年度は個別ミーティング中心とすることもあります.

研究室に配属が決まったら,遠慮せずに研究室に来てください.先輩の様子を見て,話を聞いてみてください.卒業研究において必要なものは何か,考えてください.見る・聞くだけでなく自分でやってみることも大切です.
{\sffamily\bfseries 積極性・主体性}を重視します.{\sffamily\bfseries 2年次生のうちから先輩の卒業研究を引き継ぎ,改善したものを学外で発表して受賞した先輩たち}もいます.ぜひチャレンジしてください.
\vspace{-.3cm}
\subsection*{3年次生}
\vspace{-.3cm}
\begin{itemize}
  \item 必修科目ジョイントリサーチで,卒業研究に必要な知識・技術・姿勢を身に付けます.
  \item 最新技術は日々更新されます.新しい技術を活用するには専門知識も重要ですが,基礎力を鍛えることを優先してください.
\end{itemize}
\vspace{-.3cm}
\subsection*{4年次生}
\vspace{-.3cm}
\begin{itemize}
  \item 春学期:研究計画書(初稿6月初旬,最終稿7月初旬)
  \item 秋学期:卒業論文(初稿11月中旬,最終稿12月初旬),試問会(1月下旬)
\end{itemize}
\section{配属希望者へのメッセージ}

「必要であれば何でもやってみる」気持ちで広く勉強をしてください.その上で,ここが重要だというところは,とことん深く掘り下げてください.理解した「つもり」は危険です.
\begin{itemize}
  \item 研究対象への深い興味
  \item 研究の遂行に必要な知識を探し求める姿勢
  \item 卒業研究に真剣に取り組む誠実さ
  \item 日常生活を楽しめる前向きな姿勢
\end{itemize}

新カリキュラムになり,かつ,配属人数が増えていることもあり,今後の研究室運営体制は大きく変わるかもしれません.
そうであっても,真剣に研究をしている人を力の限り応援します.

\section{面接}

\begin{itemize}
  \item メールでアポイント (9 月中を推奨) をとったうえで{\bfseries 研究室訪問}や{\bfseries 教員面談}に参加してください.
  \item 教員コンタクト(面談)が必須です.成績表・課題を持参してください.また,課題についても面談時に確認しますので準備をしておいてください.
  \item 教員コンタクト時に確認する課題の内容:「卒業研究に対する意欲,準備状況 (これまでに学習したこと),今後の計画 (これから学習したいこと),最近注目している技術・ニュース,読んだ本などについて考えをまとめる」
\end{itemize}

最後に,最も重要なことはマッチングです.面談での受け答えを重視します.とはいえ緊張し過ぎて自分を出せないのも困ります.自信を持ってあなたのことを十分に説明してください.

%
%
\end{document}
